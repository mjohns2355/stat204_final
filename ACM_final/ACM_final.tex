\documentclass[manuscript,screen]{acmart}


\IfFileExists{upquote.sty}{\usepackage{upquote}}{}
\IfFileExists{microtype.sty}{% use microtype if available
  \usepackage[]{microtype}
  \UseMicrotypeSet[protrusion]{basicmath} % disable protrusion for tt fonts
}{}
\makeatletter
\@ifundefined{KOMAClassName}{% if non-KOMA class
  \IfFileExists{parskip.sty}{%
    \usepackage{parskip}
  }{% else
    \setlength{\parindent}{0pt}
    \setlength{\parskip}{6pt plus 2pt minus 1pt}}
}{% if KOMA class
  \KOMAoptions{parskip=half}}
\makeatother

%%
%% This is file `sample-manuscript.tex',
%% generated with the docstrip utility.
%%
%% The original source files were:
%%
%% samples.dtx  (with options: `manuscript')
%% 
%% IMPORTANT NOTICE:
%% 
%% For the copyright see the source file.
%% 
%% Any modified versions of this file must be renamed
%% with new filenames distinct from sample-manuscript.tex.
%% 
%% For distribution of the original source see the terms
%% for copying and modification in the file samples.dtx.
%% 
%% This generated file may be distributed as long as the
%% original source files, as listed above, are part of the
%% same distribution. (The sources need not necessarily be
%% in the same archive or directory.)
%%
%%
%% Commands for TeXCount
%TC:macro \cite [option:text,text]
%TC:macro \citep [option:text,text]
%TC:macro \citet [option:text,text]
%TC:envir table 0 1
%TC:envir table* 0 1
%TC:envir tabular [ignore] word
%TC:envir displaymath 0 word
%TC:envir math 0 word
%TC:envir comment 0 0
%%
%%
%% The first command in your LaTeX source must be the \documentclass command.


% Options for packages loaded elsewhere
\PassOptionsToPackage{unicode}{hyperref}
\PassOptionsToPackage{hyphens}{url}
\PassOptionsToPackage{dvipsnames,svgnames,x11names}{xcolor}

\IfFileExists{bookmark.sty}{\usepackage{bookmark}}{\usepackage{hyperref}}

%% PANDOC PREAMBLE BEGINS

\usepackage{longtable,booktabs,array}
\usepackage{calc} % for calculating minipage widths
% Correct order of tables after \paragraph or \subparagraph
\usepackage{etoolbox}
\makeatletter
\patchcmd\longtable{\par}{\if@noskipsec\mbox{}\fi\par}{}{}
\makeatother
% Allow footnotes in longtable head/foot
\IfFileExists{footnotehyper.sty}{\usepackage{footnotehyper}}{\usepackage{footnote}}
\makesavenoteenv{longtable}
\usepackage{graphicx}
\makeatletter
\newsavebox\pandoc@box
\newcommand*\pandocbounded[1]{% scales image to fit in text height/width
  \sbox\pandoc@box{#1}%
  \Gscale@div\@tempa{\textheight}{\dimexpr\ht\pandoc@box+\dp\pandoc@box\relax}%
  \Gscale@div\@tempb{\linewidth}{\wd\pandoc@box}%
  \ifdim\@tempb\p@<\@tempa\p@\let\@tempa\@tempb\fi% select the smaller of both
  \ifdim\@tempa\p@<\p@\scalebox{\@tempa}{\usebox\pandoc@box}%
  \else\usebox{\pandoc@box}%
  \fi%
}
% Set default figure placement to htbp
\def\fps@figure{htbp}
\makeatother





\setlength{\emergencystretch}{3em} % prevent overfull lines

\providecommand{\tightlist}{%
  \setlength{\itemsep}{0pt}\setlength{\parskip}{0pt}}



 
\usepackage[]{natbib}
\bibliographystyle{te.bst}


\definecolor{mypink}{RGB}{219, 48, 122}
\makeatletter
\@ifpackageloaded{caption}{}{\usepackage{caption}}
\AtBeginDocument{%
\ifdefined\contentsname
  \renewcommand*\contentsname{Table of contents}
\else
  \newcommand\contentsname{Table of contents}
\fi
\ifdefined\listfigurename
  \renewcommand*\listfigurename{List of Figures}
\else
  \newcommand\listfigurename{List of Figures}
\fi
\ifdefined\listtablename
  \renewcommand*\listtablename{List of Tables}
\else
  \newcommand\listtablename{List of Tables}
\fi
\ifdefined\figurename
  \renewcommand*\figurename{Figure}
\else
  \newcommand\figurename{Figure}
\fi
\ifdefined\tablename
  \renewcommand*\tablename{Table}
\else
  \newcommand\tablename{Table}
\fi
}
\@ifpackageloaded{float}{}{\usepackage{float}}
\floatstyle{ruled}
\@ifundefined{c@chapter}{\newfloat{codelisting}{h}{lop}}{\newfloat{codelisting}{h}{lop}[chapter]}
\floatname{codelisting}{Listing}
\newcommand*\listoflistings{\listof{codelisting}{List of Listings}}
\makeatother
\makeatletter
\makeatother
\makeatletter
\@ifpackageloaded{caption}{}{\usepackage{caption}}
\@ifpackageloaded{subcaption}{}{\usepackage{subcaption}}
\makeatother
%% PANDOC PREAMBLE ENDS

\setlength{\parindent}{10pt}
\setlength{\parskip}{0pt}

\hypersetup{
  pdftitle={Reproduction of Hanabi AI},
  pdfauthor={MJ Johns; Adam; Markus},
  colorlinks=true,
  linkcolor={blue},
  filecolor={Maroon},
  citecolor={Blue},
  urlcolor={red},
  pdfcreator={LaTeX via pandoc, via quarto}}

%% \BibTeX command to typeset BibTeX logo in the docs
\AtBeginDocument{%
  \providecommand\BibTeX{{%
    Bib\TeX}}}

%% Rights management information.  This information is sent to you
%% when you complete the rights form.  These commands have SAMPLE
%% values in them; it is your responsibility as an author to replace
%% the commands and values with those provided to you when you
%% complete the rights form.
\setcopyright{acmcopyright}
\copyrightyear{2018}
\acmYear{2018}
\acmDOI{XXXXXXX.XXXXXXX}

%% These commands are for a PROCEEDINGS abstract or paper.
\acmConference[Conference acronym 'XX]{Make sure to enter the correct
conference title from your rights confirmation}{June 03--05,
2018}{Woodstock, NY}
\acmPrice{15.00}
\acmISBN{978-1-4503-XXXX-X/18/06}

%% Submission ID.
%% Use this when submitting an article to a sponsored event. You'll
%% receive a unique submission ID from the organizers
%% of the event, and this ID should be used as the parameter to this command.
%%\acmSubmissionID{123-A56-BU3}

%%
%% For managing citations, it is recommended to use bibliography
%% files in BibTeX format.
%%
%% You can then either use BibTeX with the ACM-Reference-Format style,
%% or BibLaTeX with the acmnumeric or acmauthoryear sytles, that include
%% support for advanced citation of software artefact from the
%% biblatex-software package, also separately available on CTAN.
%%
%% Look at the sample-*-biblatex.tex files for templates showcasing
%% the biblatex styles.
%%

%%
%% The majority of ACM publications use numbered citations and
%% references.  The command \citestyle{authoryear} switches to the
%% "author year" style.
%%
%% If you are preparing content for an event
%% sponsored by ACM SIGGRAPH, you must use the "author year" style of
%% citations and references.
%% Uncommenting
%% the next command will enable that style.
%%\citestyle{acmauthoryear}


%% end of the preamble, start of the body of the document source.
\begin{document}


%%
%% The "title" command has an optional parameter,
%% allowing the author to define a "short title" to be used in page headers.
\title[Hope]{Reproduction of Hanabi AI}

%%
%% The "author" command and its associated commands are used to define
%% the authors and their affiliations.
%% Of note is the shared affiliation of the first two authors, and the
%% "authornote" and "authornotemark" commands
%% used to denote shared contribution to the research.


  \author{MJ Johns}
  \orcid{0009-0002-4016-2517}
            \affiliation{%
                  \institution{University of California Santa Cruz}
                                                  \country{USA}
                      }
        \author{Adam}
  
            \affiliation{%
                  \institution{University of California Santa Cruz}
                                                  \country{USA}
                      }
        \author{Markus}
  
            \affiliation{%
                  \institution{University of California Santa Cruz}
                                                  \country{USA}
                      }
      
\renewcommand{\shortauthors}{Johns et al.}

%% By default, the full list of authors will be used in the page
%% headers. Often, this list is too long, and will overlap
%% other information printed in the page headers. This command allows
%% the author to define a more concise list
%% of authors' names for this purpose.
%\renewcommand{\shortauthors}{Trovato et al.}
%%  
%% The abstract is a short summary of the work to be presented in the
%% article.
\begin{abstract}
Reproduction and extension of an exploratory analysis of different
implementations of a game-playing AI for the board game Hanabi. The
original paper examined three implementations: a baseline called `Outer'
which has been used in past implementations, an intentional AI (called
Intentional), and an AI which combines the two (called `Full'). The
authors of the original paper found that the Intentional AI outperformed
the other two implementations when paired with a human player. Here we
reproduce their results and extend our analysis to consider whether
player perception of AI skill, intentionality, or likeability may have
influenced the score, regardless of implementation. We also perform
Machine Learning on the game logs and propose an update to the AI
framework to include predictive modeling of player behavior.    
\end{abstract}

%%
%% The code below is generated by the tool at http://dl.acm.org/ccs.cfm.
%% Please copy and paste the code instead of the example below.
%%
\begin{CCSXML}
<ccs2012>
 <concept>
  <concept_id>10010520.10010553.10010562</concept_id>
  <concept_desc>Computer systems organization~Embedded systems</concept_desc>
  <concept_significance>500</concept_significance>
 </concept>
 <concept>
  <concept_id>10010520.10010575.10010755</concept_id>
  <concept_desc>Computer systems organization~Redundancy</concept_desc>
  <concept_significance>300</concept_significance>
 </concept>
 <concept>
  <concept_id>10010520.10010553.10010554</concept_id>
  <concept_desc>Computer systems organization~Robotics</concept_desc>
  <concept_significance>100</concept_significance>
 </concept>
 <concept>
  <concept_id>10003033.10003083.10003095</concept_id>
  <concept_desc>Networks~Network reliability</concept_desc>
  <concept_significance>100</concept_significance>
 </concept>
</ccs2012>
\end{CCSXML}

\ccsdesc[500]{Computer systems organization~Embedded systems}
\ccsdesc[300]{Computer systems organization~Redundancy}
\ccsdesc{Computer systems organization~Robotics}
\ccsdesc[100]{Networks~Network reliability}

%%
%% Keywords. The author(s) should pick words that accurately describe
%% the work being presented. Separate the keywords with commas.
\keywords{datasets, neural networks, gaze detection, text tagging}


%%
%% This command processes the author and affiliation and title
%% information and builds the first part of the formatted document.
\maketitle

\setlength{\parskip}{-0.1pt}

\section{Introduction}\label{introduction}

Template: ACM's consolidated article template, introduced in 2017,
provides a consistent \LaTeX~style for use across ACM publications, and
incorporates accessibility and metadata-extraction functionality
necessary for future Digital Library endeavors. Numerous ACM and
SIG-specific \LaTeX~templates have been examined, and their unique
features incorporated into this single new template.

Our introduction\ldots{}

\section{Exploratory Data Analysis}\label{exploratory-data-analysis}

Discuss our EDA methods\ldots{}

ANOVA / Tukey

\section{Methods}\label{methods}

Study description

Quantitative Analysis

\section{Results}\label{results}

What do we find\ldots our reproduction supports the original author's
conclusion about the Intentional AI

\section{Discussion and Conclusion}\label{discussion-and-conclusion}

We conclude that\ldots{}

\begin{itemize}
\tightlist
\item
\end{itemize}

The ACM Reference Format text is required for all articles over one page
in length, and is optional for one-page articles (abstracts).

\section{CCS Concepts and User-Defined
Keywords}\label{ccs-concepts-and-user-defined-keywords}

The ACM Computing Classification System ---
\url{https://www.acm.org/publications/class-2012} --- is a set of
classifiers and concepts that describe the computing discipline. Authors
can select entries from this classification system, via
\url{https://dl.acm.org/ccs/ccs.cfm}, and generate the commands to be
included in the \LaTeX~source.

\section{Tables}\label{tables}

The ``\texttt{acmart}'' document class includes the
``\texttt{booktabs}'' package --- \url{https://ctan.org/pkg/booktabs}
--- for preparing high-quality tables.

Table captions are placed \emph{above} the table.

Because tables cannot be split across pages, the best placement for them
is typically the top of the page nearest their initial cite. To ensure
this proper ``floating'' placement of tables, use the environment
\textbf{table} to enclose the table's contents and the table caption.
The contents of the table itself must go in the \textbf{tabular}
environment, to be aligned properly in rows and columns, with the
desired horizontal and vertical rules. Again, detailed instructions on
\textbf{tabular} material are found in the \LaTeX~User's Guide.

Immediately following this sentence is the point at which
Table~\ref{tbl-freq} is included in the input file; compare the
placement of the table here with the table in the printed output of this
document.

\begin{longtable}[]{@{}ccl@{}}
\caption{Frequency of Special Characters}\label{tbl-freq}\tabularnewline
\toprule\noalign{}
Non-English or Math & Frequency & Comments \\
\midrule\noalign{}
\endfirsthead
\toprule\noalign{}
Non-English or Math & Frequency & Comments \\
\midrule\noalign{}
\endhead
\bottomrule\noalign{}
\endlastfoot
Ø & 1 in 1,000 & For Swedish names \\
\(\pi\) & 1 in 5 & Common in math \\
\$ & 4 in 5 & Used in business \\
\(\Psi^2_1\) & 1 in 40,000 & Unexplained usage \\
\end{longtable}

\section{Acknowledgments}\label{acknowledgments}

Identification of funding sources and other support, and thanks to
individuals and groups that assisted in the research and the preparation
of the work should be included in an acknowledgment section, which is
placed just before the reference section in your document.

This section has a special environment:

\begin{verbatim}
  \begin{acks}
  ...
  \end{acks}
\end{verbatim}

so that the information contained therein can be more easily collected
during the article metadata extraction phase, and to ensure consistency
in the spelling of the section heading.

Authors should not prepare this section as a numbered or unnumbered
\texttt{\textbackslash{}section}; please the
``\texttt{\textbackslash{}acks}'' environment.

%% begin pandoc before-bib
%% end pandoc before-bib
%% begin pandoc biblio
%% end pandoc biblio
%% begin pandoc include-after
%% end pandoc include-after
%% begin pandoc after-body
%% end pandoc after-body

\end{document}
\endinput
%%
%% End of file `sample-manuscript.tex'.
